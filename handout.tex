\documentclass[12pt, a4paper]{article}
\usepackage[utf8]{inputenc}
\usepackage[T1]{fontenc} % enable italics in titles
\usepackage{graphicx}
\setlength{\marginparwidth}{2cm}
\usepackage[a4paper,left=2.5cm,right=3.5cm,top=2.5cm,bottom=2cm,bindingoffset=0mm]{geometry}

% pdf inclusion
\usepackage{pdfpages}

% inline todo
\usepackage{todonotes}
\newcommand{\todoinline}[1]{\todo[inline]{#1}}

% nice indicator function and mathbb/mathfrak
\usepackage{bbm}
\usepackage{amssymb}

% text in math mode
\usepackage{amsmath}

% theorems
\usepackage{amsthm}
\newtheorem{theorem}{Theorem}[section]
\newtheorem{corollary}[theorem]{Corollary}
\newtheorem{lemma}[theorem]{Lemma}
\theoremstyle{definition}
\newtheorem{definition}[theorem]{Definition}
\theoremstyle{remark}
\newtheorem*{remark}{Remark}

% algorithms
\usepackage[linesnumbered, ruled, noline, noend]{algorithm2e}

% bibliography
\usepackage[
    backend=biber,
    style=alphabetic,
    sortlocale=de_DE,
    natbib=true,
    url=false, 
    doi=true,
    eprint=false
]{biblatex}
\addbibresource{literature.bib}

% proper pdf linking
\usepackage{hyperref}
\hypersetup{
    colorlinks=true,
    linkcolor=red,
	citecolor=blue,    
    urlcolor=blue,
    bookmarksopen=true,
    pdftitle={asdf}
}
\usepackage{bookmark}

% easy math commands
\newcommand{\N}{\ensuremath{\mathbb{N}}}
\newcommand{\NinN}[1][n]{\ensuremath{#1 \in \N}}
\newcommand{\Ninf}[1][n]{\ensuremath{#1 \rightarrow \infty}}
\newcommand{\X}{\ensuremath{\mathcal{X}}}
\newcommand{\R}{\ensuremath{\mathbb{R}}}
\newcommand{\RZero}{\ensuremath{\R_{\ge 0}}}
\newcommand{\Rbar}{\ensuremath{\overline{\R}}}
\newcommand{\ELL}[1]{\ensuremath{\mathcal{L}^{#1}}}
\newcommand{\K}{\ensuremath{\mathbb{K}}}
\newcommand{\indic}[2][]{\ensuremath{\mathbbm{1}_{#2}#1}}
\newcommand{\M}[2][]{\ensuremath{\mathcal{M}_{#1}(#2)}}
\newcommand{\PM}[1]{\ensuremath{\mathcal{P}(#1)}}
\newcommand{\TP}[2]{\ensuremath{\Pi(#1, #2)}}
\newcommand{\push}[2][T]{\ensuremath{{#1} \# #2}}
\newcommand{\MF}[3][]{\ensuremath{\mathfrak{M}_{#1}(#2, #3)}}
\newcommand{\inner}[2]{\ensuremath{\langle #1, #2 \rangle}}
\newcommand{\conj}[1]{\ensuremath{\overline{#1}}}
\newcommand{\map}[3][\varphi]{\ensuremath{#1:#2\rightarrow#3}}
\newcommand{\weak}[2]{\ensuremath{#1 \rightharpoonup #2}}
\newcommand{\leb}[2][]{\ensuremath{\lambda^{#1}(#2)}}
\newcommand{\supp}[1]{\ensuremath{\text{supp}(#1)}}
\newcommand{\CF}[2][]{\ensuremath{C_{#1}(#2)}}
\newcommand{\D}[1][\mu]{\ensuremath{\text{\normalfont{}d}{#1}}}
\newcommand{\Dx}[2][x]{\ensuremath{\text{\normalfont{}d}{#2}({#1})}}
\newcommand{\CConc}[1]{\ensuremath{c\text{-}conc(#1)}}
\newcommand{\CBConc}[1]{\ensuremath{\bar{c}\text{-}conc(#1)}}
\newcommand{\Orlicz}[2][\Omega]{\ensuremath{L^{#2}(#1)}}
\newcommand{\LLL}[1][\Omega]{\ensuremath{L \log L (#1)}}
\newcommand{\Lexp}[1][\Omega]{\ensuremath{L_{\exp}(#1)}}
\newcommand{\norm}[1]{\left\lVert#1\right\rVert}
\DeclareMathOperator*{\argmin}{arg\,min}
\DeclareMathOperator*{\argmax}{arg\,max}

% figures
\graphicspath{{images/}}
\renewcommand{\thefigure}{\arabic{figure}}

% title
\title{A Non-Conforming Dual Approach for Adaptive Trust-Region Reduced Basis Approximation of PDE-Constrained Parameter Optimization}
\author{Peter Oehme}
\date{24 June 2022}

\begin{document}

\maketitle
\pagenumbering{arabic}
\pagestyle{plain}

% abstract
\begin{abstract}
    \todoinline{Write abstract for handout}
\end{abstract}

\todoinline{Think about overall introductory paragraph}

% section 1
\section{Overview}

\subsection{Problem setting}

The method under investigation in~\cite{Keil2021} considers the following problem: Given a rectangular parameter domain
\begin{equation*}\label{ParRect}
    \mathcal{P} := \{ \mu \in \mathbb{R}^P \; | \; \mu_a \leq \mu \leq \mu_b \} \subseteq \mathbb{R}^P,
\end{equation*}
where the inequalities are to be considered componentwise, we consider for a real-valued Hilbert space $V$ the optimization problem
\begin{equation}\label{SettingOpti}
    \min\limits_{(u, \mu) \in V \times \mathcal{P}} \mathcal{J}(u, \mu) \quad s.t. \quad a_\mu(u, v) = l_\mu(v) \quad \forall v \in V.
\end{equation}

Here $a_\mu$ is a continuous, coercive, bilinear form, and $l_\mu$ is a continuous, linear functional; and $\mathcal{J}$ is a quadratic cost functional defined by the continuous, symmetric, bilinear form $k_\mu$, a continuous, linear functional $j_\mu$, and some parameter function $\Theta$ through
\begin{equation*}
    \mathcal{J}(u, \mu) := k_\mu(u, u) + j_\mu(u) + \Theta(\mu).
\end{equation*}
The constraint $a_\mu(u, v) = l_\mu(v) \; \forall v \in V$ encodes that $u$ must be a solution of the underlying PDE in variatonal form.

To tackle this problem in a repeatable manner, an adaptive trust-region method is employed on a reduced basis model working with a non-conforming dual (NCD) basis approach.
The advantages of this procedure are:
\begin{itemize}
    \item higher accuracy of approximations in contrast to conforming methods, and
    \item increased performance when compared to other model reduction approaches.
\end{itemize}

\subsection{Structure}

In the Section~\ref{sec:Preliminaries} we introduce the individual parts required to understand the contents of the paper~\cite{Keil2021}.
We introduce the theory of reduced basis methods from a background in finite elements, distinguishing G\^{a}teaux and Fr\'{e}chet differentiability to state the main objective of PDE constrained optimization, and motivate the two main optimization frame works used to implement the final algorithm.
Afterwards, Section~\ref{sec:ROMOptSys} highlights the differences between the standard approach to optimality systems in reduced order modelling and the NCD-corrected approach, concluding with an overview of the \textit{a posteriori} error analysis necessary to show a certified approximation property.
The adaptive trust-region reduced basis algorithm researched by~\cite{Qian2017, Keil2021} is lastly stated in Section~\ref{sec:AdapTRRBAlg} with a short addendum of convergence proofs.
The proofs of two selected error estimates can be found in Appendix~\ref{sec:Appendix}.

% section 2
\section{Preliminaries}

\todoinline{Write introductory paragraph}

\subsection{Reduced Basis Method}

This section introduces the reduced basis method (RBM) motivated by a consideration of parametrized PDEs with a large number of parameters.
This might be the case in optimization, parameter identification or other similar multi-query scenarios where the repeated computation might otherwise be too expensive.
A quick overview of RBM the reduced basis method will be given here, while its advantages, limitations, and challenges can be found in works such as~\cite{Ohlberger2015},~\cite{Quarteroni2015} or~\cite{Hesthaven2016}.

We start off with a generic description of the finite element method.
The central problem for this method is to find some $u \in V$ such that the following equation may be satisfied
\begin{equation}\label{FOMEq}
    a_\mu(u, v) = l_\mu(v) \qquad \forall v \in V,
\end{equation}
where $a_\mu$ is a continuous, coercive, bilinear form and $l_\mu$ is a continuous linear functional on some function space $V$.
In general, $V$ is supposed to be a high dimensional function space used to obtain a model with negligible error when compared to the true solution.
This model is called the full order model (FOM).
We note here that the resulting system of equations is far too expensive to compute for mutliple parameter values in sequence, motivating the introduction of RBM.\@

The first step is to only consider a certain $N$-dimensional subspace $V_N \subseteq V$ of the FOM space, where $N << \dim(V)$.
As a result the problem we want to solve is to find some $u_N \in V_N$ such that
\begin{equation}\label{ROMEq}
    a_\mu(u_N, v) = l_\mu(v) \qquad \forall v \in V_N,
\end{equation}
is satisfied.
Thus far, the complexity to solve this problem for multiple parameters has been reduced, however we can do even better.
The key assumption to acheive this is the so-called \textbf{parameter-separability}.
This means that we can decompose $a_\mu$ and $l_\mu$ in the following manner
\begin{equation}\label{ParamSep}
    a_\mu(u, v) = \sum\limits_{p = 1}^{p_a} \theta_p^a(\mu) \, a_p(u, v), \qquad l_\mu(v) = \sum\limits_{p = 1}^{p_l} \theta_p^l(\mu) \, l_p(v),
\end{equation}
where the $\theta$ are parameter functions for $a_\mu$ and $l_\mu$ respectively, the $p$ are the number of parameters for $a_\mu$ and $l_\mu$ respectively, and the $a_p$ and $l_p$ are the parameter independent components for $a_\mu$ and $l_\mu$ respectively.

With this assumption in hand, we can introduce the idea which enables RBM to repeatedly compute the solution for different parameters: the \textbf{offline/online decomposition}.
The principal idea is to first assemble the parameter independent system of equations and later only compute the parameter dependent parts for each parameter.
In this procedure, the computationally complex assembly is done only once (\textbf{offline phase}), while the computationally faster addition of parameters can be cheaply repeated multiple times (\textbf{online phase}).
We thus assemble
\begin{align*}\label{OffOnComp}
    &\mathbb{A}^{i, j}_m := a_m(\varphi_j, \varphi_i), &&\mathbb{L}^i_n := l_n(\varphi_i), \tag*{(offline)} \\
    &\mathbb{A}(\mu) := \sum\limits_{p = 1}^{p_a} \theta_p^a(\mu)\, \mathbb{A}_p, &&\mathbb{L}(\mu) := \sum\limits_{p = 1}^{p_l} \theta_p^l(\mu)\, \mathbb{L}_p, \tag*{(online)}
\end{align*}
for some basis $\{ \varphi_i \; | \; 1 \leq i \leq N \}$ of the space $V_N$, $1 \leq m \leq p_a$, and $1 \leq n \leq p_l$.
We can afterwards obtain our solution by solving the system
\begin{equation*}\label{OnOffSystem}
    \mathbb{A}(\mu)\, \underbar{u}_N(\mu) = \mathbb{L}(\mu), \qquad u_N(\mu) = \sum\limits_{i = 1}^N \underbar{u}_N(\mu)\, \varphi_i.
\end{equation*}

Overall this approach via offline/online decomposition requires a complexity of $\mathcal{O}(N^2 \, p_a) + \mathcal{O}(N \, p_l) + \mathcal{O}(N^3) + \mathcal{O}(N)$, where the first two terms amount to the offline assembly, the third summand originates from solving the assembled system, and the linear term is due to the reassembly of the solution from the solution vector.

One essential question in RBM is the construction of the reduced space $V_N$. For this there are multiple methods such as:
\begin{itemize}
    \item Greedy algorithm: construction of the reduced basis by choosing the parameter solution with the largest \textit{a posteriori} error, cf.~\cite{DeVore2013, Veroy2003},
    \item Proper orthogonal decomposition: construction of the reduced basis by left singular values, cf.~\cite{Kunisch2001, Haasdonk2008}, and
    \item Discrete empirical interpolation: construction of the reduced basis by finding a unisolvent set for an interpolation operator, cf.~\cite{Barrault2004, Carlberg2011, Chaturantabut2010, Drohmann2012}.
\end{itemize}

\subsection{PDE Constrained Optimization}

We now want to introduce the idea of optimization under PDE constraints.
This type of optimization is similar to other kinds of optimization in that we can derive certain optimality conditions reliant on the gradient and Hessian of some function or functional.
When working with PDEs as constraints, we often have to deal with objective functionals requiring that we take care in describing how the derivatives of these functionals are defined and calculated.
Hence we first give the definitions of \textbf{G\^{a}teaux} and \textbf{Fr\'{e}chet differentiability} in accordance with~\cite[Section 1.4]{Hinze2009}.

A functional $F: X \rightarrow Y$ is called G\^{a}teaux differentiable at $x \in X$ if the directional derivative $dF(x): X \rightarrow Y, h \mapsto dF (x) [ h ]$ is a bounded and linear functional, that is $dF(x) \in \mathcal{L}(X, Y)$, where the directional derivative of $F$ at $x$ is defined by
\begin{equation*}\label{DirDeriv}
    dF(x)[h] := \lim\limits_{t \rightarrow 0} \frac{F(x + th) - F(x)}{t} \in Y.
\end{equation*}
If furthermore the approximation
\begin{equation*}\label{FrechDeriv}
    \norm[Y]{F(x + th) - F(x) - dF(x)[h]} = o(\norm[X]{h})
\end{equation*}
holds for $\norm[X]{h} \rightarrow 0$, we say that $F$ is Fr\'{e}chet differentiable at $x$.
The usual generalization to say that $F$ is G\^{a}teaux/Fr\'{e}chet differentiable if it is G\^{a}teaux/Fr\'{e}chet differentiable at every $x \in X$ also applies here.

\todoinline{Write paragraph about different notions of differentiability}

\todoinline{Write paragraph about PDE opt}

\subsection{Trust-Region Methods}

\todoinline{Write paragraph about TRM}

% section 3
\section{ROM Optimality Systems}

\subsection{Standard Approach for a ROM Optimality System}

\todoinline{Write paragraph for standard approach}

\subsection{NCD-corrected Approach for a ROM Optimality System}

\todoinline{Write paragraph for ncd approach}

\subsection{\textit{A posteriori} error analysis}

\todoinline{Write paragraph for a posteriori errors~\cite{Keil2021}}

% section 4
\section{Adaptive Trust-Region Reduced Basis Algorithm and Convergence}

\subsection{The TR-RB Algorithm}

\begin{frame}{TR-RB Algorithm}
    \begin{block}{Choice of Model Function}
        We choose
        \begin{equation*}
            m^{(k)}(\eta) := \mathcal{J}_N^{(k)}(\mu^{(k)} + \eta),
        \end{equation*}
        where $\eta \in \mathcal{P}$. \\~\\

        \only<2>{Note: $\mathcal{J}_N \neq \mathcal{J}_N^{(k)}$! These are based upon different spaces $V_N^{pr, k}, V_N^{du, k}$.}
    \end{block}
\end{frame}

\begin{frame}{TR-RB Algorithm}
    \only<1>{
        \begin{algorithm}[H]
            \While{not converged}{
                Compute locally optimal solution\;
                
                \If{some condition}{
                    Continue with refinement\;
                }
                \ElseIf{some other condition}{
                    Repeat optimization step with smaller domain\;
                }
                \ElseIf{some third condition}{
                    Refine and possibly repeat optimization step\;
                }
            }
        \end{algorithm}
    }

    \only<2>{
        \begin{algorithm}[H]
            \While{not converged}{
                Compute locally optimal solution\;
                
                \If{\color{red}$\mathcal{J}_N^{(k)}(\mu^{(k + 1)}) + \Delta_{\mathcal{J}_N^{(k)}}(\mu^{(k + 1)}) < \mathcal{J}_N^{(k)}(\mu^{(k)})$}{
                    Continue with refinement\;
                }
                \ElseIf{some other condition}{
                    Repeat optimization step with smaller domain\;
                }
                \ElseIf{some third condition}{
                    Refine and possibly repeat optimization step\;
                }
            }
        \end{algorithm}
    }

    \only<3>{
        \begin{algorithm}[H]
            \While{not converged}{
                Compute locally optimal solution\;
                
                \If{\color{red}parameter smaller}{
                    Continue with refinement\;
                }
                \ElseIf{some other condition}{
                    Repeat optimization step with smaller domain\;
                }
                \ElseIf{some third condition}{
                    Refine and possibly repeat optimization step\;
                }
            }
        \end{algorithm}
    }

    \only<4>{
        \begin{algorithm}[H]
            \While{not converged}{
                Compute locally optimal solution\;
                
                \If{parameter smaller}{
                    Continue with refinement\;
                }
                \ElseIf{\color{red}$\mathcal{J}_N^{(k)}(\mu^{(k + 1)}) + \Delta_{\mathcal{J}_N^{(k)}}(\mu^{(k + 1)}) > \mathcal{J}_N^{(k)}(\mu^{(k)})$}{
                    Repeat optimization step with smaller domain\;
                }
                \ElseIf{some third condition}{
                    Refine and possibly repeat optimization step\;
                }
            }
        \end{algorithm}
    }

    \only<5>{
        \begin{algorithm}[H]
            \While{not converged}{
                Compute locally optimal solution\;
                
                \If{parameter smaller}{
                    Continue with refinement\;
                }
                \ElseIf{\color{red}parameter larger}{
                    Repeat optimization step with smaller domain\;
                }
                \ElseIf{some third condition}{
                    Refine and possibly repeat optimization step\;
                }
            }
        \end{algorithm}
    }

    \only<6>{
        \begin{algorithm}[H]
            \While{not converged}{
                Compute locally optimal solution\;
                
                \If{parameter smaller}{
                    Continue with refinement\;
                }
                \ElseIf{parameter larger}{
                    Repeat optimization step with smaller domain\;
                }
                \Else{
                    Refine and possibly repeat optimization step\;
                }
            }
        \end{algorithm}
    }

    \only<7>{
        \begin{algorithm}[H]
            \While{not converged}{
                Compute locally optimal solution\;
                
                \If{parameter smaller}{
                    \color{red} Refine and potentially enlarge trust-radius\;
                }
                \ElseIf{parameter larger}{
                    \color{red} Repeat optimization step with smaller domain\;
                }
                \Else{
                    \color{red} Refine and check whether to repeat optimization (shrinking, enlarging, or keeping of trust-radius depending on outcome)\;
                }
            }
        \end{algorithm}
    }
\end{frame}

\subsection{Convergence of the TR-RB Algorithm}

\begin{frame}{Overview on Convergence}
    % todo highlight the three different approaches from keil, qian, banholzer
\end{frame}

\begin{frame}{Convergence Arguments}
    asdf~\cite{Qian2017}
\end{frame}

\begin{frame}{Convergence Arguments}
    qwre~\cite{Keil2021}
\end{frame}

\begin{frame}{Convergence Arguments}
    zxcv~\cite{Banholzer2020}
\end{frame}

\printbibliography{}

\end{document}