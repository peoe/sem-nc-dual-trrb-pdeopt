\section{Preliminaries}

\todoinline{Write introductory paragraph}

\subsection{Reduced Basis Method}

This section introduces the reduced basis method (RBM) motivated by a consideration of parametrized PDEs with a large number of parameters.
A quick overview of RBM the reduced basis method will be given here, while its advantages, limitations, and challenges can be found in works such as~\cite{Ohlberger2015}, \cite{Quarteroni2015} or~\cite{Hesthaven2016}.

The main cause for RBM is the fast computation of PDEs for multiple parameter values.
This might be an objective in optimization, parameter identification or other similar multi-query scenarios where the repeated computation might otherwise be too expensive.
The central constraint for these methods is to find some $u \in V$ such that the following equation may be satisfied
\begin{equation}\label{StateEq}
    a_\mu(u, v) = l_\mu(v) \qquad \forall v \in V,
\end{equation}
where $a_\mu$ is a continuous, coercive, bilinear form and $l_\mu$ is a continuous linear functional on some function space $V$.
The specific choice of $V$ does not play a role for the general introduction of RBM but it will matter later on.

A key assumption is that of \textbf{parameter-separability}.
This means that we can decompose $a_\mu$ and $l_\mu$ in the following manner
\begin{equation}\label{ParamSep}
    a_\mu(u, v) = \sum\limits_{i = 1}^{p_a} \theta_i^a(\mu) \, a_i(u, v), \qquad l_\mu(v) = \sum\limits_{i = 1}^{p_l} \theta_i^l(\mu) \, l_i(v),
\end{equation}
where the $\theta$ are parameter functions for $a_\mu$ and $l_\mu$ respectively, the $p$ are the number of parameters for $a_\mu$ and $l_\mu$ respectively, and the $a_i$ and $l_i$ are the parameter independent components for $a_\mu$ and $l_\mu$ respectively.

With this assumption in hand, we can introduce the idea which enables RBM to repeatedly compute the solution for different parameters: the \textbf{offline-online-decomposition}.
The principal idea is to first assemble the parameter independent system of equations and later only compute the parameter dependent parts for each parameter.
In this procedure, the computationally complex assembly is done only once (\textbf{offline phase}), while the computationally faster \textit{parametrization} can be repeated multiple times (\textbf{online phase}).
\todoinline{look up specifics for offline-online decomp!}
\todoinline{look up better description for parametrization!}
\todoinline{add computational complexities here!}

\todoinline{Write paragraph about basis selection: greedy, POD, DEIM!}

\subsection{Fr\'{e}chet and Gateaux Differentiablility}

\todoinline{Write paragraph about different notions of differentiability}

\subsection{PDE Constrained Optimization}

\todoinline{Write paragraph about PDE opt}

\subsection{Trust-Region Methods}

\todoinline{Write paragraph about TRM}