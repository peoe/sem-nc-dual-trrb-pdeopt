\section{Overview}

\subsection{Problem setting}

The method under investigation in~\cite{Keil2021} considers the following problem: Given a rectangular parameter domain
\begin{equation*}\label{ParRect}
    \mathcal{P} := \{ \mu \in \mathbb{R}^P \; | \; \mu_a \leq \mu \leq \mu_b \} \subseteq \mathbb{R}^P,
\end{equation*}
where the inequalities are to be considered componentwise, we consider for a real-valued Hilbert space $V$ the optimization problem
\begin{equation}\label{SettingOpti}
    \min\limits_{(u, \mu) \in V \times \mathcal{P}} \mathcal{J}(u, \mu) \quad s.t. \quad a_\mu(u, v) = l_\mu(v) \quad \forall v \in V.
\end{equation}

Here $\mathcal{J}$ is a quadratic cost functional defined by the continuous, symmetric, bilinear form $k_\mu$, a continuous, linear functional $j_\mu$, and some parameter function $\Theta$ through
\begin{equation*}
    \mathcal{J}(u, \mu) := k_\mu(u, u) + j_\mu(u) + \Theta(\mu);
\end{equation*}
$a_\mu$ is a continuous, coercive, bilinear form; and $l_\mu$ is a continuous, linear functional.

To tackle this problem in a repeatable manner, an adaptive trust-region method is employed on a reduced basis model working with a non-conforming dual basis approach.
The advantages of this procedure are:
\begin{itemize}
    \item higher accuracy of approximations in contrast to conforming methods, and
    \item increased performance when compared to other model reduction approaches.
\end{itemize}

\subsection{Structure}

In the Section~\ref{sec:Preliminaries} we introduce the individual parts required to understand the contents of the paper~\cite{Keil2021}.
We introduce the theory of reduced basis methods from a background in finite elements, differentiate G\^{a}teaux and Fr\'{e}chet differentiability to state the main objective of PDE constrained optimization, and motivate the two main optimization frame works used to implement the final algorithm.
Afterwards, Section~\ref{sec:ROMOptSys} highlights the differences between the standard approach to optimality systems in reduced order modelling and the NCD-corrected approach, concluding with an overview of the \textit{a posteriori} error analysis necessary to show a certified approximation property.
The adaptive trust-region reduced basis algorithm researched by~\cite{Qian2017, Keil2021} is lastly stated in Section~\ref{sec:AdapTRRBAlg} with a short addendum of convergence proofs.
The proofs of two selected error estimates can be found in Appendix~\ref{sec:Appendix}.